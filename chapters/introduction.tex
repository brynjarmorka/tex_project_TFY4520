\chapter{Introduction}
\label{chap:introduction}

%
%
% \section{Main goal}
% Main goal
The main goal of this project is to improve EDS analysis.
There are multiple ways to improve EDS analysis, which can be qualitative, quantitative, or both.
One way is to make the analysis more transparent, which would make it easier to understand and use.
A second option is to improve the input parameters of the analysis by control checking the instrument with a known sample.
A third way is to make the quantitative analysis more accurate, which would improve EDS analysis.
In this project, the main focus have been trying to improve the transparency of the analysis. Thus, the problem statement was formulated as:


\begin{mainprob}
    \label{mainproblem}
    \textbf{How do different spectroscopy data processing influence the qualitative analysis, and how do they affect the quantitative Cliff-Lorimer analysis in HyperSpy?}
\end{mainprob}

% explain the problem statement
Most of the time used in this project was spent on trying to understand some of the methods of data processing in EDS analysis.
The different methods were applied on the same data sets to see how they  influence the qualitative analysis.
Towards the end the different methods were applied on the data before the Cliff-Lorimer quantification in HyperSpy to see how they influence the quantitative analysis.
Solving the main problem statement with an open-source Jupyter Notebook would increase the transparency of the analysis and allow users to both adjust their analysis and understand better the analysis.
The main problem statement was broken down into five sub-problems.
The sub-problems and a short description follow below.


\begin{subprob} \label{subproblem1}
    How accurate is the out-of-the-box quantification in AZtec?
\end{subprob}
\begin{subprob} \label{subproblem2}
    What are done with the data at the different steps in the analysis when using HyperSpy?
\end{subprob}
\begin{subprob} \label{subproblem3}
    How are the peaks and the background be modelled?
\end{subprob}
\begin{subprob} \label{subproblem4}
    How is the spectrum calibrated, and is AZtec different from HyperSpy?
\end{subprob}
% \begin{subprob} \label{subproblem5}
%     How does different background models affect the quantitative analysis done in HyperSpy?
% \end{subprob}
\begin{subprob} \label{subproblem6}
    When does the analysis fail, both in AZtec and HyperSpy?
\end{subprob}


% enumerated list
% \begin{subprob}
%     \begin{enumerate}
%         \item How accurate it the out-of-the-box quantification in AZtec and HyperSpy?
%         \item What are done with the data at the different steps in the analysis when using HyperSpy?
%         \item How can the peaks and the background be modelled in a way that is easy to understand?
%         \item How is the spectrum calibrated, and is AZtec different than HyperSpy?
%         \item How does different background models affect the quantitative analysis done in HyperSpy?
%         \item When does the analysis fail, both in AZtec and HyperSpy?
%     \end{enumerate}
% \end{subprob}

%
%
% \section{EDS analysis today}
\ton{Should I have a short paragraph here about the status of EDS analysis today?}



%%%%% REWRITE THIS PARAGRAPH %%%%%
Paragraph about why EDS in AZtec is for dummies.
% Transparency: why EDS in AZtec is for dummies.
% One of the main problems with the existing software for EDS analysis is that they are like black boxes.
% The manufacturer of EDS sensors, like EDAX, Hitatchi, Thermo Fisher Scientific, and Oxford Instruments \ton{Cite this?}, provide their own software for analyzing EDS data.
% These software packages are black boxes, because the code is hidden, and the users does not know what is happening inside the software.
% In other words, the user pushes some buttons to start the analysis of their sample, and then the software does some "magic" to analyze the data and produce some results.
% The user has few options to change the analysis to fit their needs.
% Many users tend to accept the results \ton{Need to cite something here?} from the software without questioning them, even though the analyzation "magic" differs between software packages and might be unreliable.
% The manufacturer Hitachi have trouble with separating Cs from ??, and their solution is to neglect the existence of Cs \brynjar{Find citation}.
% The manufacturer Oxford Instruments provides the software AZtec which have trouble with ??, and their solution is ?? \brynjar{Find something, eg. zero peak}.
% Even if the software is not wrong, the user might not understand the results, and the user have no way to change the analysis to fit their needs.
% Some might say that using e.g. AZtec is EDS analysis for dummies.
% One could solve this problem with an increase in transparency, because that would make it easier to understand EDS analysis and easier for users to adapt the analysis to their own needs.



% OBS! Goldstein har "Creating a QC Project", side 229 (QC = quality control), og DTSA-II har funksjonalitet for QC. Der skriver de bare at det er viktig med en fast prøve og noen bestemte målegreier.

\brynjar{Paragraph about Dispersion, offset, energy resolution?}

\brynjar{Paragraph about Other parameters of EDS analysis?}

% Den gamle standardtesten er veldig utdatert. Noen må finne et alternativ for parameterne til en detektor, og hva testprøven skal være. Finnes det noe standard noe sted som noen bruker allerede (ikke bruk mye tid på å finne ut om noen andre har en standard)

\brynjar{Paragraph about Improving quantitative EDS analysis?}

% structure of the report
This remainder of this report is built up around the main problem statement and the sub-problems.
The theory chapter contains the physics of X-rays and empirical adjustments in the analysis, a section on data processing, and a section about the hardware in an EDS setup.
The method chapter explains how the data was collected, while the arguments for and against the different methods are presented in the discussion chapter.
The result chapter contains qualitative and quantitative results.
The qualitative results are presented as figures of spectra showing the elements in the sample, and also how well different calibrations fit with the theoretical values.
The quantitative results are presented as tables with compositional results with different methods and adjustments.
The different methods are using AZtec and two approaches in HyperSpy.
The different adjustments are results with different calibrations, different background models, \brynjar{"and different peak models"?}
The discussion chapter follow the structure of the sub-problems, and discuss both the methods and the results of the analysis consecutively.
The conclusion chapter summarizes the report with an answer to the main problem statement, and provides ideas for further work.
The appendix contains the code used in the analysis, which is also available on GitHub \brynjar{Link to GitHub}.