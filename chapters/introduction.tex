\chapter{Introduction}
\label{chap:introduction}

%
%
% \section{Main goal}
% Main goal
The main goal of this project is to improve EDS analysis. There are multiple ways to do this.
One could make the analysis more transparent, which would make it easier to understand and use.
One could also improve the input parameters of the analysis by checking the instrument with a known sample.
Another way is to make the quantitative analysis more accurate, which would improve EDS analysis.
In this project, the focus began with enhancing transparency, then moved to improving input parameters, and finally ended with trying to improve accuracy.


Problem statement \dots

%
%
% \section{EDS analysis today}
EDS analysis today \dots

% \begin{itemize}
%     \item EDS in AZtec is for dummies.
%     \item What the user understands.
%     \item Cliff-Lorimer for thin sample.
%     \item The reason for and problem with k-factors. (and k-ratios)
% \end{itemize}


% Transparency: why EDS in AZtec is for dummies.
The manufacturer of EDS sensors, like EDAX, Hitatchi, Thermo Fisher Scientific, and Oxford Instruments \ton{Cite this?}, provide software for analyzing EDS data.
These software packages are black boxes, because the code is hidden, and the users does not know what is happening inside the software.
In other words, the user pushes some buttons to start the analysis of their sample, and then the software does some "magic" to analyze the data and produce some results.
The user has few options to change the analysis to fit their needs.
Many users tend to accept the results \ton{Need to cite something here?} from the software without questioning them, even though the analyzation "magic" differs between software packages and might be unreliable.
The manufacturer Hitachi have trouble with separating Cs from ??, and their solution is to neglect the existence of Cs \brynjar{Find citation}.
The manufacturer Oxford Instruments provides the software AZtec which have trouble with ??, and their solution is ?? \brynjar{Find something, eg. zero peak}.
Even if the software is not wrong, the user might not understand the results, and the user have no way to change the analysis to fit their needs.
Some might say that using e.g. AZtec is EDS analysis for dummies.
One could solve this problem with an increase in transparency, because that would make it easier to understand EDS analysis and easier for users to adapt the analysis to their own needs.




% What the user understands

%
%
% \section{Parameters of EDS analysis}
Parameters of EDS analysis

% Den gamle standardtesten er veldig utdatert. Noen må finne et alternativ for parameterne til en detektor, og hva testprøven skal være. Finnes det noe standard noe sted som noen bruker allerede (ikke bruk mye tid på å finne ut om noen andre har en standard)

% OBS! Goldstein har "Creating a QC Project", side 229 (QC = quality control), og DTSA-II har funksjonalitet for QC. Der skriver de bare at det er viktig med en fast prøve og noen bestemte målegreier.

Dispersion, offset, energy resolution \dots


%
%
% \section{Improving quantitative EDS analysis}
Improving quantitative EDS analysis

