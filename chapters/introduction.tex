\chapter{Introduction}
\label{chap:introduction}

%
%
% \section{Main goal}
% Main goal
The main goal of this project is to improve EDS analysis. There are multiple ways to do this.
One way is to make the analysis more transparent, which would make it easier to understand and use.
A second option is to improve the input parameters of the analysis by control checking the instrument with a known sample.
A third way is to make the quantitative analysis more accurate, which would improve EDS analysis.
In this project, the main focus have been trying to improve the transparency of the analysis. Thus, the problem statement was formulated as:

\begin{quote}
    \textbf{How can the transparency of the steps in the EDS analysis be improved, so that the analysis is easier to understand and use?
    }
\end{quote}


% sub-problems 
With this problem statement, the following sub-problems were formulated:

\begin{itemize}
    \item How accurate are the out-of-the-box quantification in AZtec and HyperSpy?
    \item What are done with the data at the different steps in the analysis?
    \item How is the spectrum calibrated, and is AZtec different than HyperSpy?
    \item How can the peaks and the background be modelled in a way that is easy to understand?
    \item How does different background models affect the quantitative analysis done in HyperSpy?
    \item When does the analysis fail, both in AZtec and HyperSpy?
\end{itemize}



%
%
% \section{EDS analysis today}
\ton{Should I have a short paragraph here about the status of EDS analysis today?}


% Transparency: why EDS in AZtec is for dummies.
One of the main problems with the existing software for EDS analysis is that they are like black boxes.
The manufacturer of EDS sensors, like EDAX, Hitatchi, Thermo Fisher Scientific, and Oxford Instruments \ton{Cite this?}, provide their own software for analyzing EDS data.
These software packages are black boxes, because the code is hidden, and the users does not know what is happening inside the software.
In other words, the user pushes some buttons to start the analysis of their sample, and then the software does some "magic" to analyze the data and produce some results.
The user has few options to change the analysis to fit their needs.
Many users tend to accept the results \ton{Need to cite something here?} from the software without questioning them, even though the analyzation "magic" differs between software packages and might be unreliable.
The manufacturer Hitachi have trouble with separating Cs from ??, and their solution is to neglect the existence of Cs \brynjar{Find citation}.
The manufacturer Oxford Instruments provides the software AZtec which have trouble with ??, and their solution is ?? \brynjar{Find something, eg. zero peak}.
Even if the software is not wrong, the user might not understand the results, and the user have no way to change the analysis to fit their needs.
Some might say that using e.g. AZtec is EDS analysis for dummies.
One could solve this problem with an increase in transparency, because that would make it easier to understand EDS analysis and easier for users to adapt the analysis to their own needs.




% What the user understands

%
%

% OBS! Goldstein har "Creating a QC Project", side 229 (QC = quality control), og DTSA-II har funksjonalitet for QC. Der skriver de bare at det er viktig med en fast prøve og noen bestemte målegreier.

\brynjar{Dispersion, offset, energy resolution?}


\brynjar{Other parameters of EDS analysis?}

% Den gamle standardtesten er veldig utdatert. Noen må finne et alternativ for parameterne til en detektor, og hva testprøven skal være. Finnes det noe standard noe sted som noen bruker allerede (ikke bruk mye tid på å finne ut om noen andre har en standard)


%
%
% \section{Improving quantitative EDS analysis}
\brynjar{Improving quantitative EDS analysis?}

