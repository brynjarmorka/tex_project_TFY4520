\chapter{Results}

% Qualitative first, as is the way in EDS.
% 	- spectrum calibrated, 5-30 kV
% 	- one plot for all
% 	- Look at energy scale for all results. How does the peaks deviate with the Ga As calibration?
% 	- Cu tape (give type) is not good as a Cu reference.
% 	- The stub is not Fe as expected, but Al. Makes sence, since steel is alloys and is magnetic.

% Quantitative
% 	- initial
% 	- background
% 	- peak fitting


\section{Introduction}
\label{sec:results:intro}
The results are presented in this chapter.
First qualitative then quantitative results are presented.


\section{Qualitative results}
\label{sec:results:qualitative}


\subsection{Calibration}
\label{sec:results:qualitative:calibration}

% dispersion and offset table
% initial quantification
% chapter Results
\begin{table}[ht]
    \centering
    \caption{
        Different calibration values.
        The AZtec calibration is reffered to as the uncalibrated value.
        The dispersion is calculated with \cref{eq:theory:calibration:dispersion}.
        The offset is calculated with \cref{eq:theory:calibration:offset}.
        The own calibration was done on Ga L$_\alpha$ and As K$_\alpha$ from the 30 kV measurement on the GaAs wafer.
    }
    \label{tab:results:calibrations}
    \begin{tabular}{m{4cm} m{2cm} m{2cm}}
        Calibration method & Dispersion,   & Zero offset \\
                           & [keV/channel] & [channels]  \\
        \hline
        AZtec              & 50 \%         & 50 \%       \\
        HyperSpy           & 50 \%         & 50 \%       \\
        Own calibration    & 50 \%         & 50 \%       \\
        30 kV              & 50 \%         & 50 \%
    \end{tabular}
\end{table}


% calibration peak accuracy table
% table with peak accuracy

\begin{table}[ht]
    \centering
    \caption{
        Peak accuracy of the different calibration methods.
        The accuracy here is the deviation from the theoretical peak to the measured peak.
        The measured peak is the Gaussian fitted center of the peak.
        All results are from the 30 kV measurements.
        The own calibration was done on Ga L$_\alpha$ and As K$_\alpha$ from the 30 kV measurement on the GaAs wafer.
        The HyperSpy cailbration was done by making a model and fitting it to the data on the 30 kV GaAs spectrum.
        AZ is short for AZtec.
        HS is short for HyperSpy.
        All deviations are in percentage difference from the theoretical peak value.
    }
    \label{tab:results:calibration-peak-accuracy}
    \begin{tabular}{ccccc}
        Peak and line & Theoretical [keV] & AZ deviation [\%] & HS deviation [\%] & Own deviation [\%] \\
        \hline
        As L$\alpha$  & 1.2819            & 1.000             & 0.439             & 0.422              \\
        As K$\alpha$  & 10.5436           & -0.202            & -0.025            & -0.010             \\
        Ga L$\alpha$  & 1.098             & 1.044             & 0.342             & 0.318              \\
        Ga K$\alpha$  & 9.2517            & -0.153            & 0.009             & 0.024              \\
        Cu L$\alpha$  & 0.9295            & 1.767             & 0.888             & 0.857              \\
        Cu K$\alpha$  & 8.0478            & -0.114            & 0.031             & 0.045              \\
        Mo K$\alpha$  & 17.4793           & -0.325            & -0.108            & -0.090             \\
        Mo L$\alpha$  & 2.2932            & 1.047             & 0.859             & 0.858              \\
        Si K$\alpha$  & 1.7397            & 0.167             & -0.175            & -0.182             \\
        Al K$\alpha$  & 1.4865            & 0.200             & -0.247            & -0.259             \\
        Cu K$\alpha$  & 8.0478            & -0.116            & 0.029             & 0.043              \\
        C K$\alpha$   & 0.2774            & -2.955            & -6.583            & -6.738
    \end{tabular}
\end{table}



\section{Quantitative results}
\label{sec:results:quantification}


\subsection{Initial quantification}
\label{sec:results:initial_quantification}
% How accurate it the out-of-the-box quantification in AZtec and HyperSpy?

The initial quantification was done on the data from the GaAs wafer in AZtec and in HyperSpy as out-of-the-box as possible.
The results are presented in \cref{tab:initial_quantification}.
The wafer is a 1:1 alloy of gallium and arsenic, so the atomic percent of Ga and As should be 50\% and 50\% respectively.

% initial quantification
% chapter Results
\begin{table}[ht]
    \centering
    \caption{
        Initial quantification of the GaAs wafer.
        The ratio in the wafer is 1:1, so the correct ratio is 50\% and 50\%, because the results are in atomic percent.
    }
    \label{tab:initial_quantification}
    \begin{tabular}{m{1.5cm} m{1.5cm} m{1.5cm} m{1.5cm} m{1.5cm}}
        $V_\textnormal{acc}$ & AZtec &       & HyperSpy &       \\
                             & Ga    & As    & Ga       & As    \\
        \hline
        5 kV                 & 50 \% & 50 \% & 50 \%    & 50 \% \\
        10 kV                & 50 \% & 50 \% & 50 \%    & 50 \% \\
        15 kV                & 50 \% & 50 \% & 50 \%    & 50 \% \\
        30 kV                & 50 \% & 50 \% & 50 \%    & 50 \%
    \end{tabular}
\end{table}
