\chapter{Results}
\label{chap:results}

% Qualitative first, as is the way in EDS.
% 	- spectrum calibrated, 5-30 kV
% 	- one plot for all
% 	- Look at energy scale for all results. How does the peaks deviate with the Ga As calibration?
% 	- Cu tape (give type) is not good as a Cu reference.
% 	- The stub is not Fe as expected, but Al. Makes sense, since steel is alloys and is magnetic.

% Quantitative
% 	- initial
% 	- background
% 	- peak fitting


% \section{Introduction}
% \label{sec:results:intro}
The results are presented in this chapter.
First qualitative then quantitative results are presented.
All the spectra taken were qualitatively analyzed.
Only the GaAs bulk spectra was quantitatively analyzed.
The spectrum from the GaAs bulk wafer taken on 30 kV is shown in \cref{fig:GaAs30kV_HS}.
This plot was made with HyperSpy, which utilize Matplotlib for plotting.
The plotting method in HyperSpy can add where the theoretical peak centers are.
The lines added also show an estimate of the weight of the peak.


% add figure figures/GaAs30kV_HS.png
\begin{figure}[h]
    \centering
    \includegraphics[width=0.70\textwidth]{figures/GaAs30kV_HS.png}
    \caption{
        The GaAs spectrum taken at 30 kV.
        % This plot was made with HyperSpy, which use Matplotlib.
        % The theoretical peak centers are added as lines.
        (a) is the whole spectrum.
        (b) is the zoomed in on the L-peaks.
        (c) is zoomed in on the K-peaks.
        (d) is zoomed in on the Ga K$\alpha$ peak.
        This plot has the calibration from AZtec, and it is clear that the line position is deviating from the center of the peaks.
    }
    \label{fig:GaAs30kV_HS}
\end{figure}




\section{Qualitative results}
\label{sec:results:qualitative}

This qualitative section present first the spectra from each sampled area, then some general remarks and finally the calibration of the spectra. The spectra are presented as plots and the calibrations are presented as tables.
% Use time on this to show that I have results and that I know what I am doing.
% It will also help to point at errors in AZtec.


\subsection{Each sample area}
\label{sec:results:qualitative:each_sample_area}

\cref{fig:results:Spectra_Al} to \cref{fig:results:Spectra_NW} shows the spectra for the six different areas of the sample plotted with Plotly.
The plots are available as an interactive HTML plots on the GitHub repository.
\brynjar{Upload the HTML files to the GitHub repository.}
The calibration used in these spectra is based on the calibration of Ga L$\alpha$ and As K$\alpha$ from the GaAs bulk wafer.
The y-axis is normalized to the highest peak value in each spectrum, i.e. the highest peak is always 1. % discuss: why normalize.
The spectra are scaled to show the region of interest for each sample area.
Peaks which are taller than the y-axis are marked with a gray stippled line.
The annotated energy on the plots are where the annotation line ends, and it not necessarily exactly the center of the peak when the peak is Gaussian fitted and nor exactly the theoretical line value.
Even though the annotated values deviate a few eV, the goal in the qualitative analysis is to figure out what the peaks are.
\brynjar{Move to discussion?}



% figure Spectra_Al.png
\begin{figure}[h!]
    \centering
    \includegraphics[width=0.90\textwidth]{figures/each_spectra/Al_everything.png}
    \caption{
        The spectrum of the FIB stub, which gave a strong Al signal.
        This was expected to be Fe, but very little Fe signal was found.
        The tallest peak is at 1.49 keV, and it is the Al K$\alpha$ peak with some signal at the K$\beta$ peak at 1.56 keV.
        The relative weight for Al K$\beta$ to K$\alpha$ is 0.013 (from HyperSpy).
        Fe K$\alpha$ at 6.40 keV has a question mark, because the FIB stub was initially expected to be Fe.
        The signal from Fe K$\alpha$ is barely a signal different from the background.
    }
    \label{fig:results:Spectra_Al}
\end{figure}

% The Al spectra
\cref{fig:results:Spectra_Al} shows the spectrum of the FIB stub, which gave a strong Al signal.
Only one spectrum was taken of the FIB stub, and it was taken at 30 kV.
As for all the other spectra, this spectrum have a zero peak, a C K$\alpha$ peak at 0.260 keV and a O K$\alpha$ peak at 0.51 keV.
Most of the other spectra also have a Si K$\alpha$ peak at 1.74 keV.
The tallest peak with a relative intensity of 1 is the combined peak of Al K$\alpha$ and K$\beta$ at 1.49 keV.
The contribution of Al K$\beta$ to Al K$\alpha$ is 0.013, which is small, but still changes the peak shape.
The FIB stub was initially expected to be Fe, but the spectrum lacks a Fe K$\alpha$ peak at 6.40 keV.
There are some signal at 6.40 keV, but it is barely a signal different from the background.
The peaks at 1.25 keV and 5.9 keV indicate that the FIB stub is an alloy of mostly Al with some Mg and Mn.
% discussion: Al-Mg-Mn alloy, but unable to quantify with my bad signal, or?
The last peak in the spectrum is the Al K$\alpha$ sum peak at 2.97 keV. % discussion: what other peaks could this be.
\ton{When stating that some peaks are sum peaks, do you want me to state what other possibilities they could be, or can I just discuss that in the discussion?}
The background is increase rapidly and almost linearly from C K$\alpha$ to Al K$\alpha$, and drops to 10\% after the Al K$\alpha$ peak.
\ton{How much of the free text do you want me to repeat in the figure text?}



% The Si spectra
\cref{fig:results:Spectra_Si} shows the spectra taken from the pure Si wafer sample part.
As in the other spectra, this spectrum have a zero peak, a C K$\alpha$ peak at 0.260 keV and a O K$\alpha$ peak at 0.51 keV.
In addition, there is an unidentified peak at 0.080 keV.
The tallest peak with a relative intensity of 1 is the Si K$\alpha$ peak at 1.73 keV, which have a contribution from the K$\beta$ peak at 1.84 keV.
The relative weight for Si K$\beta$ to K$\alpha$ is 0.028.
As in the Ai spectrum, the Si spectra have a sum peak from the tallest peak.
The sum peak is at 3.48 keV, which is the sum peak of the Si K$\alpha$.
In all four spectra the background drops significantly after the Si K$\alpha$ peak.
The relative drop down is biggest in the 30 kV spectrum.

% figure Spectra_Si.png
\begin{figure}[h]
    \centering
    \includegraphics[width=0.90\textwidth]{figures/each_spectra/Si_everything.png}
    \caption{
        The spectra of the pure Si wafer sample part.
        All four spectra have one large peak at 1.73 keV, which is the Si K$\alpha$ peak with some signal at the K$\beta$ peak at 1.83 keV.
        The relative weight for Si K$\beta$ to K$\alpha$ is 0.028.
        The zero peak is marked at 0.008 keV.
        After the zero peak there is another sharp peak at 0.080 keV, which is not identified.
        The energies annotated are the end of the annotation line, which can deviate a few percent from the actual peak energy.
    }
    \label{fig:results:Spectra_Si}
\end{figure}


% The Cu spectra
\cref{fig:results:Spectra_Cu} shows the spectra taken from the Cu-tape sample part.
These spectra have a zero peak, a C K$\alpha$, a O K$\alpha$ peak and a small Si K$\alpha$ peak.
The tallest peak with a relative intensity of 1 is the C K$\alpha$ peak at 0.26 keV.
The Cu K$\alpha$ and Cu K$\beta$ peaks are only visible at the 30 kV spectrum.
The height of the Cu K$\alpha$ peak is 0.017, which means that the C K$\alpha$ peak is more than 55 times taller.
None of the spectra have a Cu L$\alpha$ peak, which should have been at 0.93 keV.
The 30 kV spectrum have a signal at 2.29 keV, which could be from Mo L$\alpha$, but no Mo K$\alpha$ signal is visible.
% However, the 30 kV spectrum have no peak at the Mo K$\alpha$ peak at 17.48 keV. % discussion: The other spectra show that the L peaks are taller. But still no singal except noise at all at 17.48 keV.

% figure Spectra_Cu.png
\begin{figure}[h]
    \centering
    \includegraphics[width=0.90\textwidth]{figures/each_spectra/Cu_everything.png}
    \caption{
        The spectra of the Cu sample part.
        The Cu sample was Cu-tape from the lab, but the Cu K$\beta$ peak is only barely visible at the 30 kV spectrum.
        The highest peak in all three spectra is at 0.260 keV, which is the C K$\alpha$ peak, slightly off from the expected 0.277 keV.
        The plot is limited to 9.5 keV, because there are no peaks above that energy.
        The Mo L$\alpha$ peak at 2.29 keV is only visible in the 30 kV spectrum.
        The Mo L$\alpha$ is marked with a question mark, because there are no Mo K$\alpha$ signal.
    }
    \label{fig:results:Spectra_Cu}
\end{figure}


% figure Spectra_Mo.png
\begin{figure}[h!]
    \centering
    \includegraphics[width=0.90\textwidth]{figures/each_spectra/Mo_everything.png}
    \caption{
        The spectra of the Mo sample part.
        The Mo K$\alpha$ peak at 17.47 keV is the 30 kV spectrum, with a high noise level.
        The high double peak is Mo L$\alpha$ at 2.29 keV and Mo L$\beta$ at 2.39 keV.
        The Mo L$\beta$1 peak at 2.49 keV has the weight 0.327.
        In the Mo spectra the Si K$\alpha$ line is just barely visible as a very small peak, and is not annotated.
        The peak at 2.01 keV is the Mo Ll and the peak at 2.83 keV is the Mo M$\gamma$3.
        Both are with name and energy from the HyperSpy database.
        Mo Ll and Mo L$\gamma$3 have weights at 0.041 and 0.011, respectively.
    }
    \label{fig:results:Spectra_Mo}
\end{figure}

% The Mo spectra
\cref{fig:results:Spectra_Mo} shows the spectra taken from the Mo-disk sample part.
The four spectra have a zero peak, a C K$\alpha$, a O K$\alpha$ peak.
The Si K$\alpha$ give a very small signal, and is visible as the tiny peak before Mo Ll.
The background of the 5 kV spectrum is very high, but that is due do its lower signal on the Mo L$\alpha$, making the 0.18 keV peak the tallest and thus scaling up the background.
The tallest peak in the 5 kV spectrum is the 0.18 keV peak, which is also visible in the other three spectra.
The energy of 0.18 keV only match with the B K$\alpha$ line at 0.1833 keV. % discuss, it is really this peak? The accuracy is low at low E, use the talbe which show that C Ka miss a lot.
The tallest peak in the 10, 15 and 30 kV spectra is the Mo L$\alpha$ peak at 2.29 keV, which is contributed by the Mo L$\beta$1 peak at 2.39 keV.
The HyperSpy database have the energy and weight of two other Mo peaks visible in the spectrum, which are the Mo Ll and Mo L$\gamma$3 peaks at 2.01 keV and 2.83 keV.
The X-ray Booklet does not include these lines, but include some other lines which are not visible in these spectra.
Only the 30 kV spectra have a signal at the Mo K$\alpha$ and Mo K$\beta$ peaks at 17.46 and 19.62 keV. % discuss: as expected
The peak at 4.6 keV is the sum peak of Mo L$\alpha$ and Mo L$\beta$1.
As in the other spectra, the background drops significantly after the tallest peak.




% the GaAs spectra, with two plots
\cref{fig:results:Spectra_GaAs} shows the spectra taken from the GaAs sample part, and \cref{fig:results:Spectra_GaAs_bg_and_sum_peaks} shows a zoomed in part of the spectra to enlarged the background and the sum peaks.
All the spectra have a zero peak, a C K$\alpha$ peak, and an O K$\alpha$ peak.
The Ga Ll peak at 0.96 keV is visible in all four spectra.
The Si K$\alpha$ peak is small in all the spectra, but strongest in the 30 kV spectrum and weakest in the 5 kV spectrum.
The tallest peak in all four spectra is the Ga L$\alpha$ peak at 1.1 keV.
The As L$\alpha$ peak has a decreasing relative intensity from 5 to 30 kV.
For the first 2 keV the background is relatively highest in the 5 kV spectrum and decreases with increasing voltage.
After 2 keV this trend reverses, and the background is highest in the 30 kV spectrum. % discussion: overvoltage, singal to noise ?
The K peaks are visible in the 15 and 30 kV spectra.
The sum peaks of the L peaks are visible in the 5, 10 and 15 kV spectra, but not in the 30 kV spectrum.
The sum peaks of the K peaks are only visible in the 30 kV spectrum. % discuss: no sum peaks in the 15 kV spectrum even though the K-peaks are there. Or, the counts go from ~2 til ~6 in the area of the peaks.


% figure Spectra_GaAs.png
\begin{figure}[p] % p is to get their own page for floats
    \centering
    \includegraphics[width=0.90\textwidth]{figures/each_spectra/GaAs_everything.png}
    \caption{
        The spectra of the GaAs wafer sample part.
        This is bulk GaAs, where the ratio of Ga to As is 1:1.
        Both the K-peaks and the L-peaks of Ga and As are visible.
        There is a peak at 0.51 keV, which is the O K$\alpha$ peak.
        There is a peak at 0.24 keV, which is the C K$\alpha$ peak.
    }
    \label{fig:results:Spectra_GaAs}
\end{figure}


% figure Spectra_GaAs.png
\begin{figure}[p]
    \centering
    \includegraphics[width=0.90\textwidth]{figures/each_spectra/GaAs_bg_and_sum_peaks.png}
    \caption{
        A zoomed in view of the GaAs sample part.
        The background is easier to see, and the sum peaks are visible.
        The sum peaks are the sum of the K$\alpha$ and peaks, and the L$\alpha$ peaks.
    }
    \label{fig:results:Spectra_GaAs_bg_and_sum_peaks}
\end{figure}


% The NW spectra
\cref{fig:results:Spectra_NW} shows the spectra of the nanowire sample part.
This is the sample part with the most peaks, and contains signal from C, O, Ni, Cu, Ga, As, Si, Mo and Sb.
One of the peaks is at 0.389 keV, which could be N K$\alpha$ peak, but it could also be other elements. % Ti Ll, but no Ti K$\alpha$ peak
The Ni signal is both from the L$\alpha$ at 0.85 keV and the K$\alpha$ at 7.49 keV.
The Sb signal is from the L$\alpha$ at 3.6 keV, L$\beta$1 at 3.8 keV, and Sb L$\beta$1 at 4.1 keV. % discuss: the 3.6 peak could be smt else like ??, but the peak series show that is is Sb
The Mo signal is both from the L$\alpha$ at 2.29 keV and the K$\alpha$ at 17.47 keV, but the K peak is very weak and thus not included in the plot.
The Cu signal is both from the L$\alpha$ at 0.92 keV and the K$\alpha$ at 8.05 keV.
The tallest peak in all four spectra is the Ga L$\alpha$ peak at 1.1 keV.
In the 5 kV spectrum the C K$\alpha$ peak at 0.26 keV is equally high as the Ga L$\alpha$ peak.
The As and Ga signals and their ratios are very similar to the GaAs spectrum signal, but the sum peaks are not visible in the NW spectra.
The Ga and As L$\alpha$ sum peaks could have been visible, but the sum peak signal coincides with the Mo L$\alpha$ peak at 2.2 keV. % discussion: the sum peaks could be missing due to lower DT.
The 10, 15 and 30 kV signal have and unidentified peak at 2.00 keV.
The unidentified peak could be a sum peak or another artifact.
\ton{Any ideas for the 2.00 keV peak? It does not match well with any peaks that I've found.}

% figure Spectra_NW.png
\begin{figure}[h]
    \centering
    \includegraphics[width=0.90\textwidth]{figures/each_spectra/NW_everything.png}
    \caption{
        The spectra of the nanowire sample part.
        This spectrum has the most peaks, and contains  C, O, Ni, Cu, Ga, As, Si, Mo and Sb, and maybe N.
        The line at 0.389 keV could be N K$\alpha$ peak, but it could also be other elements. % Ti Ll, but no Ti K$\alpha$ peak. discussion: what is this peak?
        The peak at 0.92 keV which is labeled as Cu K$\alpha$ is also getting a contribution from the Ga Ll peak, at 0.95 keV. % True???
        The 30 kV spectrum had a small signal from the Mo K$\alpha$ peak at 17.47 keV, but that signal was weak.
        The 10, 15 and 30 kV signal have an unidentified peak at 2.00 keV. % what is this ???
    }
    \label{fig:results:Spectra_NW}
\end{figure}

%
%
%
%
%
%
%
%
%
% \subsection{REWRITE THIS PART!: Peaks and background}
% \label{sec:results:qualitative:peaks}
% Peaks, 
% zero peak
% peak width higher at higher E
% very strange that Cu 30 kV have a small Cu Ka and no Cu La at all (since La is bigger than Ka and Ga and Cu is close in the table), which is the case for Ga
%
%
%
%
%
%

\subsection{General results from the spectra}
\label{sec:results:qualitative:general}

All the spectra have peaks with high peak-to-background ratio. %discuss: sampling good enough? I do not know if i miss some peaks, but the ones i have are nice for qualitative analysis at least.
% The zero peak from the Oxford detector is visible in all the spectra as the first peak, with a center at 0.00924 keV.
In all the spectra, the highest peak is below 5 keV. % discussion: what peaks give the highest counts, with theory and empirical data. detector efficiency? Absorption? Overvoltage? etc
The GaAs, NW and Mo spectra show clearly that the peaks broaden with higher E, since they have peaks at low and middle to high energy. % discuss why this happens.
The width of the peaks are quantified with FWHM in the quantitative section below.
When doing the qualitative analysis, it became clear that the FIB stub was not made of Fe as expected, but rather of Al with a peak at 1.48 keV.
All the 5 kV spectra decrease more or less linearly from 1 to 5 keV. % discusiion: overvoltage
% Another discovery was that the Cu-tape does not give a good Cu signal.
% The high peak in the Cu-tape spectra at 0.260 keV are from C and not the Cu L$\alpha$ peak.
% Only the Cu-tape taken at 30 kV has a Cu peak, but it is very small.
Even though the Cu spectra at 30 kV has the Cu K$\alpha$ peak, the Cu L$\alpha$ peak is completely missing. % this is very, very strange. ! TODO: Discuss
\brynjar{Add transition sentence.}

% double peaks, ie peaks which are overlapping
Some peaks in the spectra are overlapping, which shifts the shape of the peaks.
An example of this is the As K$\alpha$ peak and the Ga K$\beta$ peak in the NW and the GaAs bulk wafer spectra.
These peaks are overlapping, but also far enough apart that the peaks are still distinguishable.
Another example of overlapping peaks is the Mo L$\alpha$ peak and the Mo L$\beta_1$, which are overlapping so much that they are hard to distinguish.
Since they are harder to distinguish, the peak fitting makes one Gaussian for the two peaks, which is off on both peak centers. % for discussu\ion: which is why the deviation of the Mo L$\alpha$ peak is off in the calibration accuracy table, \cref{tab:results:calibration-peak-accuracy}.
Overlapping peaks makes counting the signal from specific peaks harder.
\brynjar{Figure of double peaks?}

% background, 
% bg increase with lower kV, 
% bg shape, 
The signal from the background is another factor which makes counting more difficult.
In general, background in the acquired spectra is low, but with different shapes.
In the GaAs, Si and Mo the height of the background decrease with higher acceleration voltage.
The background in the Cu spectra increase with higher acceleration voltage.
The most similar background signal over different voltages are in the NW spectra. % discuss: thin sample -> less background?
The values of the background radiation is in general very low and almost flat above the highest peak in the spectra. % discussion: nothing to reduce the speed of the e-
% For example, the Cu 10 kV spectrum have its highest peak at 0.5 keV, and the background is almost zero above 0.5 keV. not the best example
The Al spectrum background is high before the high Al K$\alpha$ peak, and much lower after the high peak.
Both the value and the shape of the background is different before and after the peak.
The same behavior is clearly true in the Si spectra.
The peak where the background change is the highest peak, Si K$\alpha$.
The background values in Si 30 kV are 10 times higher before than after the highest peak.
The background shape in Si 30 kV is almost linear from 0.6 to 1.6 keV, drops to 10\% height from 1.6 to 1.9 keV, and then follows the expected background shape from 1.9 keV.
The expected background shape is illustrated in \brynjar{Make a drawing of the background.} % discussion: explain that the bg is dependent on a peak (ie material) being present. This makes fiting hard, since there are some peak-dependency which is hard to predict.
%discussion: model the peak as more than one polygon added together, since the shape changes. For further works. That is called spline.
All the other spectra show the same behavior with their highest peak and the peaks effect on the background as the Al and Si spectra.
In general, the background signals are low, but their different shapes and heights makes it harder to fit the peaks of the characteristic X-ray lines.
\brynjar{Figure of background? And figure of fit of background before and after a tall peak?}
\ton{The last sentence is meant to be a transition/finishing sentence, but might be too much discussion.} % discussion: make the background as a spline. But then, what to do under the peak?



% discussion: understanding the strays properly is actually helpful for the qualitative analysis, since some elements can be excluded based on the strays and some strays are only present in certain materials.
% Si stray in all spectra
% stray in NW outside beam, Mo, Cu and Sb
In addition to the characteristic peaks and the background, there are also artifacts and strays in the spectra. % discussion: lower peaks are super important, eg in the Al spectra where it is small amounts of Mg and Mn, probably. Asserting what is a stray and what is characteristic is important for quantification.
All the spectra have signal from the K$\alpha$ line of C and O, which could be contamination and oxide layers. % discussion: this is contamination. Carbon deposited from the beam, visisble on the Cu-tape images. Oxide layers? If yes, look at material properties.
The C and O signal is higher at lower acceleration voltage.
All the spectra have a Si peak at 1.74 keV. % This is the Si escape peak for some spectra, but not all.
For all but the Si spectra, the Si K$\alpha$ peak at 1.74 keV is a stray from outside the beam or the Si escape peak from the detector.
Some spectra have additional signals from elements outside the area of the main beam, like the Mo, Sb and Cu peak in the NW spectra. % pretty sure it is Sb, because of the series of peaks at 3.6, 3.8, 4.1 and 4.3. 
% discuss both interactin volume and bonus peaks from X-ray strays in the chamber/sample.
The Sb peaks in the NW spectra are at 3.60, 3.85, 4.10 and 4.35 keV, being the L$\alpha_1$, L$\beta_1$, L$\beta_2$ and L$\gamma_1$ peaks.
All four Mo spectra have a peak at 0.175 keV, which match best with B K$\alpha$ at 0.183 keV.
\brynjar{Figure of strays?}
\brynjar{Finishing sentence for the general observations.}


% Sum peaks, to discussion
% The Si spectra have a sum peak at 3.49 keV, which is the sum of Si K$\alpha$ at 1.74 keV. % discusiion: this could be Sb, but not like NW where Sb is a series of peaks. Also say something about the resolution of the detector, which is higher at lower E. And also that Sb cannot be excluded bc. of the K peak. And also that the sum peaks can be tested with low DT. And something here about the bad calibration in AZtec which might make this sum peak as a Sb peak, but I now know that it is a Si peak sum peak and not a Sb peak.
% The Al spectrum have its highest peak at 1.48 keV and a sum peak at 2.98 keV
% The Mo spectrum have a sum peak at 4.65 keV, which is the sum of Mo L$\alpha$ at 2.293 keV and Mo L$\beta_1$ at 2.395 keV. % same shape as Mo La  Mo Lb1, i.e. also Mo La+MoLa sum
% The GaAs spectrum at 30 kV have two small sum peak signals at 18.5 and 19.5 keV, while the NW spectrum at 30 kV with lower DT does not have these peaks. % discussion: results show that sum peaks are lower with lower DT

% Mo double peak
% GaAs double peak
% Si stray on Cu
% also a sum peak at Al 30 kV spectrum (look at DT)
% Si double count at 3.4 keV which could be Sn (but the E resolution could seperate them Ton thinks at this low keV and Sn would have multiple peaks, cannot exclude bc of the K peak could count with low DT to minimize double counts consequence of AZtecs bad calibration)




% qualitative results, calibration
\subsection{Calibration}
\label{sec:results:qualitative:calibration}

% what I did to calibrate the spectra
Different calibrations were explored.
The initial calibration is the one from AZtec, and is the one used in the spectra in \cref{fig:GaAs30kV_HS}.
This calibration has a left shift for the L-peaks and a right shift for the K-peaks.
The second type of calibration is the one given by the model fit in HyperSpy.
The third type is from the self-made model fit, using the distance between two high intensity and far apart peaks to calibrate the energy scale.
The third type is both calculated with the Ga L$\alpha$ and As K$\alpha$ peaks in the GaAs 30 kV spectrum, and with Mo L$\alpha$ and Mo K$\alpha$ peaks in the Mo 30 kV spectrum.

% the values
Values for the four calibrations are given in \cref{tab:results:calibrations}.
The deviations are a few percent, and the accuracy of the different calibrations give on specific peaks are given in \cref{tab:results:calibration-peak-accuracy}.
Here accuracy is the difference between the theoretical peak position and the peak center in the spectrum, given in percent.
For almost all the peaks, the deviation is greatest for the AZtec calibration.
One exception is the C K$\alpha$ peak, which deviates a lot less for the AZtec calibration. %discussion: Aztec might use the zero peak to calibrate, but that is just speculation.
The difference between the HyperSpy calibration and the self-made calibration on the GaAs and Mo spectra are small.
In the following qualitative section, the effect of the different calibrations on the spectra are explored.

% dispersion and offset table
% initial quantification
% chapter Results
\begin{table}[ht]
    \centering
    \caption{
        Different calibration values.
        The AZtec calibration is reffered to as the uncalibrated value.
        The dispersion is calculated with \cref{eq:theory:calibration:dispersion}.
        The offset is calculated with \cref{eq:theory:calibration:offset}.
        The own calibration was done on Ga L$_\alpha$ and As K$_\alpha$ from the 30 kV measurement on the GaAs wafer.
    }
    \label{tab:results:calibrations}
    \begin{tabular}{m{4cm} m{2cm} m{2cm}}
        Calibration method & Dispersion,   & Zero offset \\
                           & [keV/channel] & [channels]  \\
        \hline
        AZtec              & 50 \%         & 50 \%       \\
        HyperSpy           & 50 \%         & 50 \%       \\
        Own calibration    & 50 \%         & 50 \%       \\
        30 kV              & 50 \%         & 50 \%
    \end{tabular}
\end{table}


% calibration peak accuracy table
% table with peak accuracy

\begin{table}[ht]
    \centering
    \caption{
        Peak accuracy of the different calibration methods.
        The accuracy here is the deviation from the theoretical peak to the measured peak.
        The measured peak is the Gaussian fitted center of the peak.
        All results are from the 30 kV measurements.
        The own calibration was done on Ga L$_\alpha$ and As K$_\alpha$ from the 30 kV measurement on the GaAs wafer.
        The HyperSpy cailbration was done by making a model and fitting it to the data on the 30 kV GaAs spectrum.
        AZ is short for AZtec.
        HS is short for HyperSpy.
        All deviations are in percentage difference from the theoretical peak value.
    }
    \label{tab:results:calibration-peak-accuracy}
    \begin{tabular}{ccccc}
        Peak and line & Theoretical [keV] & AZ deviation [\%] & HS deviation [\%] & Own deviation [\%] \\
        \hline
        As L$\alpha$  & 1.2819            & 1.000             & 0.439             & 0.422              \\
        As K$\alpha$  & 10.5436           & -0.202            & -0.025            & -0.010             \\
        Ga L$\alpha$  & 1.098             & 1.044             & 0.342             & 0.318              \\
        Ga K$\alpha$  & 9.2517            & -0.153            & 0.009             & 0.024              \\
        Cu L$\alpha$  & 0.9295            & 1.767             & 0.888             & 0.857              \\
        Cu K$\alpha$  & 8.0478            & -0.114            & 0.031             & 0.045              \\
        Mo K$\alpha$  & 17.4793           & -0.325            & -0.108            & -0.090             \\
        Mo L$\alpha$  & 2.2932            & 1.047             & 0.859             & 0.858              \\
        Si K$\alpha$  & 1.7397            & 0.167             & -0.175            & -0.182             \\
        Al K$\alpha$  & 1.4865            & 0.200             & -0.247            & -0.259             \\
        Cu K$\alpha$  & 8.0478            & -0.116            & 0.029             & 0.043              \\
        C K$\alpha$   & 0.2774            & -2.955            & -6.583            & -6.738
    \end{tabular}
\end{table}









\section{Quantitative results}
\label{sec:results:quantification}

% from intro paragraph:
% The different methods are using AZtec and two approaches in HyperSpy.
% The different adjustments are results with different calibrations, different background models, \brynjar{"and different peak models"?}


% \subsection{Quantification methods}
% \label{sec:results:quantification_methods}
% How accurate it the out-of-the-box quantification in AZtec and HyperSpy?

% Its okay to only do the GaAs here.
% Could also look at NW, but not nessesary.

To do the quantitative analysis, HyperSpy needs k-factors.
The k-factors for Ga and As are given in \cref{tab:results:k-factors}.
These k-factors are from the GaAs bulk wafer, and HyperSpy have estimated them theoretically.
\ton{Shall I list the other k-factors for the other sample areas? I do not think I will use them, since I've only quantified the GaAs bulk wafer. But the other k-factors are results too. Eventually including NW data too, but I do not know that ratio.}

% k-factors table
\begin{table}[h]
    \centering
    \caption{
        K-factors for Ga and As, extracted from AZtec.
        All the k-factors are theoretically estimated.
        AZtec provides either the k-factor for the L$\alpha$ or the K$\alpha$ line, and selects automatically based on the energy of the incident electrons.
    }
    \label{tab:results:k-factors}
    \begin{tabular}{ccccc}
        V$_\textnormal{acc}$ [kV] & Line      & Ga k-factor & As k-factor & Ga / As \\
        \hline
        5                         & L$\alpha$ & 1.21        & 1.086       & 0.898   \\
        10                        & L$\alpha$ & 1.31        & 1.223       & 0.934   \\
        15                        & L$\alpha$ & 1.331       & 1.259       & 0.946   \\
        30                        & K$\alpha$ & 4.191       & 3.268       & 0.780
    \end{tabular}
\end{table}

The initial quantification was done on the data from the GaAs wafer in AZtec.
The results are presented in \cref{tab:initial_quantification}.
The wafer is a 1:1 alloy of gallium and arsenic, so the atomic percent of Ga and As should be 50\% and 50\% respectively.

% initial quantification table
\begin{table}[h]
    \centering
    \caption{
        Initial quantification of the GaAs wafer.
        The ratio in the wafer is 1:1, so the correct ratio is 50\% and 50\%, because the results are in atomic percent.
        \brynjar{The idea was to use both AZtec and HyperSpy, but it did not make sense to use HyperSpy with the discussion I have now. Thus it is only AZtec.}
    }
    \label{tab:initial_quantification}
    \begin{tabular}{ccc}
        $V_\textnormal{acc}$ & Ga    & As    \\
        \hline
        5 kV                 & 50 \% & 50 \% \\
        10 kV                & 50 \% & 50 \% \\
        15 kV                & 50 \% & 50 \% \\
        30 kV                & 50 \% & 50 \%
    \end{tabular}
\end{table}

To better understand the ratios between Ga and As, the areas under the peaks in the spectra were counted.
Table \cref{tab:results:ratios} gives the ratios between the areas under the peaks for 5, 10, 15 and 30 kV.
The table compares L$\alpha$ peaks, K$\alpha$ peaks, K$\beta$ peaks and the sum of the peaks.
The table also lists the FWHM of the peaks.

\begin{table}[ht]
    \centering
    \caption{
        Ratios of Ga and As on the GaAs wafer.
        The spectrum was calibrated with GaAs 30 kV.
        K$\beta$ at 15 kV was too low to be detected and is therefore not included in the table.
        The ratio column is Ga sum divided by As sum.
        The ratio * k column is the ratio multiplied by the respective k-factor in \cref{tab:results:k-factors}.
        The empty cells are for peaks where the was no k-factor or for the sum of the peaks, which have no center and no FWHM.
    }
    \label{tab:results:ratios}
    \begin{tabular}{ccccccccc}

        Peak                         & Peak*k & Ratio & Ga value & As value & Ga FWHM & As FWHM & Ga sum  & As sum \\
                                     &        &       & [keV]    & [keV]    & [eV]    & [eV]    &         &        \\
        \hline
                                     &        &       &          &          &         &         &         &        \\

        5 kV                         &        &       &          &          &         &         &         &        \\
        L$\alpha$                    & 1.151  & 1.282 & 1.101    & 1.288    & 74.010  & 80.921  & 75.462  & 58.844 \\
        \hline
                                     &        &       &          &          &         &         &         &        \\
        10 kV                        &        &       &          &          &         &         &         &        \\
        L$\alpha$                    & 1.349  & 1.444 & 1.100    & 1.287    & 73.841  & 80.827  & 76.222  & 52.770 \\

        \hline
                                     &        &       &          &          &         &         &         &        \\
        15 kV                        &        &       &          &          &         &         &         &        \\
        L$\alpha$                    & 1.579  & 1.669 & 1.100    & 1.287    & 73.830  & 81.137  & 77.001  & 46.146 \\
        K$\alpha$                    &        & 2.445 & 9.253    & 10.536   & 155.080 & 181.951 & 6.013   & 2.459  \\
        % K$\beta$       & 1.000 & 10.536         & 10.536         & 181.951      & 181.951      & 2.459   & 2.459  \\
        L$\alpha$+K$\alpha$          &        & 1.708 &          &          &         &         & 83.014  & 48.605 \\
        % L$\alpha$+K$\alpha$+K$\beta$ & 1.674 &            &            &          &          & 85.473  & 51.065 \\

        \hline
                                     &        &       &          &          &         &         &         &        \\
        30 kV                        &        &       &          &          &         &         &         &        \\
        L$\alpha$                    &        & 2.279 & 1.098    & 1.287    & 72.309  & 80.849  & 76.465  & 33.546 \\
        K$\alpha$                    & 1.301  & 1.678 & 9.253    & 10.542   & 157.799 & 168.238 & 58.718  & 34.994 \\
        K$\beta$                     &        & 1.603 & 10.276   & 11.736   & 171.804 & 185.034 & 8.821   & 5.503  \\
        L$\alpha$+K$\alpha$          &        & 1.972 &          &          &         &         & 135.184 & 68.540 \\
        L$\alpha$+K$\alpha$+K$\beta$ &        & 1.945 &          &          &         &         & 144.004 & 74.042
    \end{tabular}
\end{table}



One of the adjustments explored was the affect of the calibration on the quantification.
Using different the calibrations in \cref{tab:results:calibrations} gave different quantification results when using CL in HyperSpy. % discussion: Issue with using Clif-Lorimer since I have bulk and not thin sample.
The results are presented in \cref{tab:results:calibration-quantification}.
The same method was used on all spectra, but the quantification on 10 and 15 kV are obviously wrong.



% table with peak accuracy

\begin{table}[ht]
    \centering
    \caption{
        Quantification with different calibration methods.
        The quantification is done by in HyperSpy with Cliff-Lorimer method.
        The CL method is for this samples, while the GaAs wafer used here is a bulk sample.
        AZ is the AZtec calibration.
        HS is the HyperSpy calibration.
        GaAs is the calibration on the GaAs 30 kV spectrum.
        The accuracy of the quantification is the deviation from 50\%, because the sampled area is 1:1 GaAs wafer.
    }
    \label{tab:results:calibration-quantification}
    \begin{tabular}{cccccc}
        Vacc & Element & Line & AZ     & HS     & GaAs   \\
        \hline
        5    & As      & L    & 44.81  & 44.29  & 44.19  \\
        5    & Ga      & L    & 55.19  & 55.71  & 55.81  \\
        10   & As      & L    & 100.00 & 100.00 & 100.00 \\
        10   & Ga      & L    & 0.00   & 0.00   & 0.00   \\
        15   & As      & L    & 5.23   & 4.39   & 5.87   \\
        15   & Ga      & L    & 94.77  & 95.61  & 94.13  \\
        30   & As      & K    & 56.25  & 57.14  & 59.02  \\
        30   & Ga      & K    & 43.75  & 42.86  & 40.98
    \end{tabular}
\end{table}

% affect of calibration
% accuracy on quantification


% affect of peak and background modelling
% both errors and the quantification accuracy
% could model as other than Gaussian? Discusssion




% Conclusion: update dispersion and offset for the APREO?