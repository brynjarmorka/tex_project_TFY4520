\chapter{Results}

% Qualitative first, as is the way in EDS.
% 	- spectrum calibrated, 5-30 kV
% 	- one plot for all
% 	- Look at energy scale for all results. How does the peaks deviate with the Ga As calibration?
% 	- Cu tape (give type) is not good as a Cu reference.
% 	- The stub is not Fe as expected, but Al. Makes sence, since steel is alloys and is magnetic.

% Quantitative
% 	- initial
% 	- background
% 	- peak fitting


\section{Introduction}
\label{sec:results:intro}
The results are presented in this chapter.
First qualitative then quantitative results are presented.


\section{Qualitative results}
\label{sec:results:qualitative}


\subsection{Calibration}
\label{sec:results:qualitative:calibration}

% dispersion and offset table
% Calibration values
% chapter Results
\begin{table}[ht]
    \centering
    \caption{
        Different calibration values.
        The AZtec calibration is reffered to as the uncalibrated value.
        The dispersion is calculated with \cref{eq:theory:calibration:dispersion}.
        The offset is calculated with \cref{eq:theory:calibration:offset}.
        The own calibration was done on Ga L$_\alpha$ and As K$_\alpha$ from the 30 kV measurement on the GaAs wafer.
        The HyperSpy cailbration was done by making a model and fitting it to the data on the 30 kV GaAs spectrum.
    }
    \label{tab:results:calibrations}
    % \begin{tabular}{m{4cm} m{2cm} m{2cm}}
    \begin{tabular}{ccc}
        Calibration method & Dispersion,   & Zero offset \\
                           & [keV/channel] & [channels]  \\
        \hline
        AZtec              & 0.01          & 20          \\
        HyperSpy           & 0.010028      & 21.0787     \\
        Own calibration    & 0.010030      & 21.127
    \end{tabular}
\end{table}


% calibration peak accuracy table
% table with peak accuracy

\begin{table}[p]
    \centering
    \caption{
        % TODO: give the peak accuracy in ev, not percent. Or eV and percent?
        Peak accuracy of the different calibration methods on 30 kV spectra: NW, Mo, Si, Al, Cu. % TODO: reformulate
        The other acceleration voltages gave similar results.
        The accuracy here is the deviation from the theoretical peak to the measured peak.
        The percentage deviation is also given.
        The measured peak is the fitted center of the peak.
        % discussion: The Mo L$\alpha$ deviates much because the peak is not well fitted.
        % The C K$\alpha$ is fitted well, but deviates much more than all the other peaks.
        The self-made calibration was done on two spectra: GaAs and Mo. % discussion: Mo is more far apart than GaAs.
        % One was done on Ga L$\alpha$ and As K$\alpha$ from the 30 kV measurement on the GaAs wafer.
        % The other was done on the more far apart peaks Mo L$\alpha$ and Mo K$\alpha$ from the 30 kV measurement on the Mo wafer.
        The HyperSpy calibration was done on the GaAs spectrum.
        At the bottom the RMS of the deviation in the column is given.
    }
    \label{tab:results:calibration-peak-accuracy}
    \begin{tabular}{cccccc}
        Peak         & Theoretical & AZtec                 & HyperSpy              & Ga L$\alpha$ \& As K$\alpha$ & Mo L$\alpha$ \& Mo K$\alpha$ \\
                     & [keV]       & [eV]                  & [eV]                  & [eV]                         & [eV]                         \\
        \hline
        As L$\alpha$ & 1.2819      & 12.8,\,\,\,    1.0\%  & 5.6,\,\,\,    0.4\%   & 5.4,\,\,\,    0.4\%          & 7.2,\,\,\,    0.6\%          \\
        As K$\alpha$ & 10.5436     & -21.3,\,\,\,   -0.2\% & -2.7,\,\,\,   -0.0\%  & -1.1,\,\,\,   -0.0\%         & 9.8,\,\,\,    0.1\%          \\
        Ga L$\alpha$ & 1.098       & 11.5,\,\,\,    1.0\%  & 3.8,\,\,\,    0.3\%   & 3.5,\,\,\,    0.3\%          & 5.1,\,\,\,    0.5\%          \\
        Ga K$\alpha$ & 9.2517      & -14.2,\,\,\,   -0.2\% & 0.9,\,\,\,    0.0\%   & 2.2,\,\,\,    0.0\%          & 11.9,\,\,\,    0.1\%         \\
        Cu L$\alpha$ & 0.9295      & 16.4,\,\,\,    1.8\%  & 8.3,\,\,\,    0.9\%   & 8.0,\,\,\,    0.9\%          & 9.4,\,\,\,    1.0\%          \\
        Cu K$\alpha$ & 8.0478      & -9.2,\,\,\,   -0.1\%  & 2.5,\,\,\,    0.0\%   & 3.7,\,\,\,    0.0\%          & 12.1,\,\,\,    0.2\%         \\
        Mo K$\alpha$ & 17.4793     & -56.8,\,\,\,   -0.3\% & -18.8,\,\,\,   -0.1\% & -15.8,\,\,\,   -0.1\%        & 1.9,\,\,\,    0.0\%          \\
        Mo L$\alpha$ & 2.2932      & 24.0,\,\,\,    1.0\%  & 19.7,\,\,\,    0.9\%  & 19.7,\,\,\,    0.9\%         & 22.5,\,\,\,    1.0\%         \\
        Si K$\alpha$ & 1.7397      & 2.9,\,\,\,    0.2\%   & -3.0,\,\,\,   -0.2\%  & -3.2,\,\,\,   -0.2\%         & -1.0,\,\,\,   -0.1\%         \\
        Al K$\alpha$ & 1.4865      & 3.0,\,\,\,    0.2\%   & -3.7,\,\,\,   -0.2\%  & -3.9,\,\,\,   -0.3\%         & -1.9,\,\,\,   -0.1\%         \\
        Cu K$\alpha$ & 8.0478      & -9.3,\,\,\,   -0.1\%  & 2.4,\,\,\,    0.0\%   & 3.5,\,\,\,    0.0\%          & 11.9,\,\,\,    0.1\%         \\
        C K$\alpha$  & 0.2774      & -8.2,\,\,\,   -3.0\%  & -18.3,\,\,\,   -6.6\% & -18.7,\,\,\,   -6.7\%        & -17.9,\,\,\,   -6.5\%        \\
        \hline
        RMS  [eV]    &             & 14.75                 & 4.62                  & 4.55                         & 9.57
    \end{tabular}
\end{table}



\section{Quantitative results}
\label{sec:results:quantification}


\subsection{Initial quantification}
\label{sec:results:initial_quantification}
% How accurate it the out-of-the-box quantification in AZtec and HyperSpy?

The initial quantification was done on the data from the GaAs wafer in AZtec and in HyperSpy as out-of-the-box as possible.
The results are presented in \cref{tab:initial_quantification}.
The wafer is a 1:1 alloy of gallium and arsenic, so the atomic percent of Ga and As should be 50\% and 50\% respectively.

% initial quantification
% chapter Results
\begin{table}[ht]
    \centering
    \caption{
        Initial quantification of the GaAs wafer.
        The ratio in the wafer is 1:1, so the correct ratio is 50\% and 50\%, because the results are in atomic percent.
        \brynjar{Put in the actual results here. Use both HyperSpy linear and model fitted results?}
    }
    \label{tab:initial_quantification}
    \begin{tabular}{m{1.5cm} m{1.5cm} m{1.5cm} m{1.5cm} m{1.5cm}}
        $V_\textnormal{acc}$ & AZtec &       & HyperSpy &       \\
                             & Ga    & As    & Ga       & As    \\
        \hline
        5 kV                 & 50 \% & 50 \% & 50 \%    & 50 \% \\
        10 kV                & 50 \% & 50 \% & 50 \%    & 50 \% \\
        15 kV                & 50 \% & 50 \% & 50 \%    & 50 \% \\
        30 kV                & 50 \% & 50 \% & 50 \%    & 50 \%
    \end{tabular}
\end{table}
