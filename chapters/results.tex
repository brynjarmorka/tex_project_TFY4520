\chapter{Results}


\section{Introduction}
\label{sec:results:intro}
The results are presented in this chapter.
The sections follow the structure of the sub-problems in \cref{chap:introduction}.

\section{Initial quantification}
\label{sec:results:initial_quantification}
The initial quantification was done on GaAs in AZtec and in HyperSpy as out-of-the-box as possible. The results are presented in \cref{tab:initial_quantification}.

% GaAs 5, 10, 15 and 30 kV for AZtec and HyperSpy, 
% Table with initial quantification
% initial quantification
% chapter Results
\begin{table}[ht]
    \centering
    \caption{
        Initial quantification of the GaAs wafer.
        The ratio in the wafer is 1:1, so the correct ratio is 50\% and 50\%, because the results are in atomic percent.
    }
    \label{tab:initial_quantification}
    \begin{tabular}{m{1.5cm} m{1.5cm} m{1.5cm} m{1.5cm} m{1.5cm}}
        $V_\textnormal{acc}$ & AZtec &       & HyperSpy &       \\
                             & Ga    & As    & Ga       & As    \\
        \hline
        5 kV                 & 50 \% & 50 \% & 50 \%    & 50 \% \\
        10 kV                & 50 \% & 50 \% & 50 \%    & 50 \% \\
        15 kV                & 50 \% & 50 \% & 50 \%    & 50 \% \\
        30 kV                & 50 \% & 50 \% & 50 \%    & 50 \%
    \end{tabular}
\end{table}





\section{Steps in the analysis}
\label{sec:results:steps}
The first sub-problem was to find out what is done with the data at the different steps in the analysis.
\ton{Is this interesting to write about?}


\section{Calibration}
\label{sec:results:calibration}
The calibration of the

% Something about how the calibrated peak of Ga $K\alpha$ is directly on $K\alpha_1$.


\section{Peak and background modelling}
\label{sec:results:modelling}
The third sub-problem was to find out how the peaks and the background are modelled in a way that is easy to understand.


\section{Background models}
\label{sec:results:background}
The fourth sub-problem was to find out how different background models affect the quantitative analysis done in HyperSpy.


\section{Failure}
\label{sec:results:failure}
The fifth sub-problem was to find out when the analysis fails, both in AZtec and HyperSpy.
% quantification with 