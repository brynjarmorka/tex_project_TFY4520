\chapter*{Sammendrag}

Energidispersivrøntgenspektroskopi (EDS) er en mye brukt analysemetode for å identifisere den elementære sammensetningen av materialer.
Nøyaktigheten til EDS-analyser kan påvirkes av ulike faktorer, inkludert instrumentparametere og -oppsett, prøveforberedelse og ulike trinn i analyseprosessen.
I denne studien var målet å forbedre EDS-analysen gjennom utvikling av nye eller forbedrede databehandlingsteknikker.
Tilnærmingen involverte både kvalitativ og kvantitativ analyse.
Resultatene viser muligheter for å forbedre EDS-analysen, inkludert forbedret kalibrering av det brukte oppsettet, en idé om ny bakgrunnsmodellering og Jupyter Notebooks for å øke transparensen i analysen.
Selv om arbeidet utgjør et lite bidrag til feltet EDS-analyse, har arbeidet potensialet til å øke forståelsen av analyseprosessen for brukere.



I denne studien ble et skanneelektronmikroskop (SEM) brukt til å anskaffe EDS-data, som ble analysert i programvaren AZtec fra detektorprodusenten, med den åpne kildekoden HyperSpy og tilpasset Python-kode, med hjelp fra Jupyter Notebooks, SciPy, NumPy og Plotly.
Et av resultatene er et forslag om å forbedre kalibreringen av oppsettet som ble brukt, en Oxford Instruments Xmax 80 mm$^2$ detektor på en FEI SEM APREO, som er på NanoLab ved Norges teknisk-naturvitenskapelige universitet (NTNU).
Den nye kalibreringen baserer seg på røntgentopp- og bakgrunnsmodellering gjort med Python.
Forskjellen mellom den gjeldende og foreslåtte kalibreringen er å øke dispersjonen fra 10 eV/kanal til 10,03 eV/kanal, og øke nullpunktjusteringen fra 20 kanaler til 21,1 kanaler.
Den foreslåtte kalibreringen viser seg å ha et gjennomsnittlig kvadratisk avvik på 1/3 av den gjeldende kalibreringen, på de røntgentoppene som ble analysert i denne studien.



Noen av koden som ble brukt i denne studien er vedlagt i appendiks, og all koden er tilgjengelig på GitHub.
Koden kan plotte spekter, modellere karakteristiske røntgenstråletopper og bakgrunnsstråling, kalibrere energiskalaen og gjøre enkel kvantitativ analyse.
To reservoar er tilgjengelige, én for all koden som ble produsert i denne studien, og én destillert versjon med utvalgte deler av koden.
Reservoarene (repositories) inneholder også datagrunnlaget, og de er tilgjengelig primært for at en bruker skal kunne lettere forstå deler av EDS-analysen.
Reservoarene er tilgjengelig på:



\url{https://github.com/brynjarmorka/eds-analysis}

\url{https://github.com/brynjarmorka/eds-analysis-final}


