\chapter{Discussion}
\label{chap:discussion}


\section{Introduction}
\label{sec:discussion:intro}
The discussion is presented in this chapter. \dots



\section{Calibration}
\label{sec:discussion:calibration}

% Ga Ka where Ka2 is not visible
% point of the paragraph: empirical view is good enough
In \cref{fig:theory:GaAs30keV-K-lines} the calibrated peak of Ga $K\alpha$ is directly on $K\alpha_1$ and the $K\alpha_\textnormal{HyperSpy}$, the emission from $K\alpha_2$ is missing.
The energy difference is too small to differentiate the two peaks in EDS, but the peak has no left-shift, implying that the peak is not a mixture of $K\alpha_1$ and $K\alpha_2$.
The peak shape is \brynjar{calculate the shape}, which is close to a perfect Gaussian.
There are possible theoretical explainations for the lacking emission from $K\alpha_2$.
One explaination is the quantum mechanical effect called \dots, where two close lines appear as one stronger line.
This effect is not only present in EDS, but also in \dots.
Another explaination is that Ga $K\alpha_2$ could also be weaker than $K\alpha_1$, but there is no reason for this to be the case.
The take away from this is that the empirical view tends to be good enough for EDS analysis.